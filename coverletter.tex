\documentclass[12pt]{letter}

\usepackage{marvosym}
\usepackage{helvet}
\usepackage{times}
\usepackage{courier}
\usepackage{xspace}
\usepackage[T1]{fontenc}
\usepackage{hyperref}
% \usepackage{mycv}
\usepackage[top=15mm,left=30mm,right=30mm,bottom=15mm]{geometry}

\signature{Masataro Asai, Ph.D}
\address{60 Cross St. E APT224\\ Somerville 02145, MA, USA}

\def\position{Senior Staff Research Scientist} % position name
\def\company{Robot Intelligence Lab, Samsung Research America} % position name
\def\companyaddress{665 Clyde Avenue, Mountain View, CA, USA} % position name

\begin{document}


\begin{letter}{ % Address of the company you are applying to
\company\\
\companyaddress
}

\opening{To whom it may concern,}

% PARAGRAPH ONE: State the reason for the letter, name the position or
 % type of work you are applying for and identify the source from which
 % you learned of the opening.\\

% PARAGRAPH TWO: Indicate why you are interested in the position, the
% company, its products, services - above all, stress what you can do
% for the employer. If you are a recent graduate, explain how your
% academic background makes you a qualified candidate for the
% position. If you have practical work experience, point out specific
% achievements or unique qualifications. Try not to repeat the same
% information the reader will find in the resume. The purpose of this
% section is to strengthen your resume by providing details which bring
% your experiences to life.\\

% PARAGRAPH THREE: Request a personal interview and indicate your
% flexibility as to the time and place. Repeat your phone number in the
% letter. End the letter by thanking the employer for taking the time
% to consider your credentials.\\


Please find my CV enclosed with my application for the position of \position.
I am excited to find job openings that explore the future of intelligent robotic agents.

I believe the next leap in intelligent robotics
comes from the integration of \emph{high-level task planning and motion planning}.
Studies on connecting low-level sensorimotor inputs
with high-level symbolic / structured / semantic information have been gradually gaining traction.
Such integrations are becoming even more crucial given the advent of LLMs,
which have an excellent semantic understanding of symbolic inputs,
while they also possess unique weaknesses (hallucinations, rule compliance).
I have a history of bridging this boundary with advanced machine learning / statistical methods,
primarily through combining generative modeling and optimized combinatorial solvers.

As a senior scientist, I am looking forward to leading a team of colleagues
with open minds and a large shelf of ideas combining two distant disciplines of machine learning and symbolic reasoning.
While being able to publish on my own consistently,
I have also been mentoring graduates \& undergraduates
from MIT and other institutions in this topic,
which has led to successful publications,
including the one that received an \emph{outstanding paper award}.
The challenge of acquiring the right talent for this task is that
it requires the skill sets from both the neural and the symbolic backgrounds.
Few academics can prepare them for it because they themselves are
often exclusively from one background or the other.

I am going to attend Neurips 2025 in San Diego this December,
and would be delighted to meet for an interview during the conference.
Thank you for the time you took to consider this application.
I look forward to hearing from you in the near future.

\closing{Sincerely yours,}
\end{letter}

\end{document}